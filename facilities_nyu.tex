The NYU Department of Physics is part of the NYU Faculty of Arts \& Science and includes the Center for Cosmology and Particle Physics (the home of PI Hogg), the Center for Soft Matter Research, and the Center for Quantum Phenomena.
The NYU Department of Mathematics is part of the Courant Institute and includes the Center for Atmosphere Ocean Science (the home of PI Zanna).
A wide variety of sponsored
research activities take place in both Departments.
These range from
the theoretical to the pragmatic, and include a broad spectrum of
interactions with such disciplines as astronomy, biology, medicine, chemistry, and computer science.

In addition to benefiting from this research activity directly and indirectly, as members of these Departments, the PIs receive through their home Departments staff support for clerical work, post-award grant support, and for computing (expanded upon below).

\paragraph{Computing Equipment at Physics:}
The astrophysics group at NYU maintains some high-performance computers
for the PI, and substantial storage machines.
The latter contain more than 100~Tb of disk space, most of which is filled
with astrophysical imaging data.  The Physics Department maintains a
state-of-the-art controlled-environment computer room.

The University has hired a Director of Scientific Computing for the
NYU Center for Cosmology and Particle Physics.  His
responsibilities include management of the cluster and the data
servers, and overall management and supervision of the Center's
computer system.  He will oversee the computer hardware used in this
project.

A variety of computing resources are available within the Physics
Department, including UNIX workstations, laptops, laser color
printers, etc.  The Department runs a network of several hundred
desktop workstations and file servers.

\paragraph{Computing Equipment in Courant:}
Courant runs a network of several hundred desktop workstations and smaller high-performance computational servers, which will be used for testing, and analysis. All servers and
clusters are available to PI Zanna and her team.

\paragraph{Other NYU Computing Resources:}
Computational needs are also supported through the University's
Academic Computing Services, a unit of the NYU-wide Information
Technology Services offering an additional wide range of computational
resources in support of research and instruction.  These include a
variety of computing platforms, including several high-performance
multi-CPU and multi-GPU systems, and scientific software.  Consultants are available
to assist in the use of these resources.

In particular, these resources include NYU's central high-performance computing
cluster Greene. It has a measured Linpack benchmark performance of 2.088 petaflops. It
consists of 4 login nodes, 524 standard compute nodes with 192GB RAM and dual CPU sockets,
40 medium memory nodes with 384GB RAM and dual CPU sockets, and 4 large memory nodes
each with 3TB RAM and quad CPU sockets. All cluster nodes are equipped with 24-core Intel
Cascade Lake Platinum 8268 chips. Additionally, Greene is equipped with 73 compute nodes
each equipped with 4 NVIDIA RTX8000 GPUs and 10 of the nodes will be equipped with 4
V100 GPUs. The total count of CPU processing cores is 32,000 and 145TB of RAM.
Greene is currently the most powerful supercomputer in the New York metropolitan area, one of
the top ten most powerful supercomputers in higher education, and one of the top 100 greenest
(i.e., environmentally friendly) supercomputers in the world.

In addition, the NYU Center for Data Science (where PI Hogg is affiliated)
maintains some specialized high-performance machines for machine-learning
applications, some of which are relevant to the proposed project.

\paragraph{Office Space:}
The offices of the personnel will all be located in the NYU Physics and Mathematics Departments.

In addition, through the NYU Center for Data Science, PI Hogg has access to temporary studio space, work space, visitor space, and meeting space.

\paragraph{Library Resources:}
In addition to an enormous book collection, the NYU libraries hold
current subscriptions to hundreds of hardcopy and electronic journals.
It provides access to all journals relevant to this project and such
databases as MathSciNet, the Web of Science (science citation index),
and the ACM Digital Library.

The NYU Libraries also does some storage and preservation of research data.

\paragraph{Experimental and Hardware Facilities:}
Both the Physics Department and Math Department have experimental facilities, including machining, imaging, and clean facilities, but none of these are directly relevant to this project, except insofar as they are part of the rich intellectual and research atmosphere.
